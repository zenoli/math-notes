\subsection{Notions of Symmetry}

\newcommand{\relsymbol}{\rho}
\newcommand{\reluniverse}{U}
\newcommand{\rel}[3][\relsymbol]{#2\mathrel{#1}#3}
\newcommand{\nrel}[3][\cancel{\relsymbol}]{#2\mathrel{#1}#3}

\begin{definition}[Notions of Symmetry]

  Let $U \neq \emptyset$ and $\relsymbol \subseteq \reluniverse^2$ be a relation on $U$.
  \begin{enumerate}
    \item $\relsymbol \text{ is asymmetric } :\Leftrightarrow
            \forall x, y \in \reluniverse: \rel{x}{y} \rightarrow \nrel{y}{x}$
          \label{def:asymmetry}
    \item $\relsymbol \text{ is anti-symmetric } :\Leftrightarrow
            \forall x, y \in \reluniverse: \rel{x}{y} \land \rel{y}{x} \rightarrow x = y$
          \label{def:anti-symmetry}
    \item $\relsymbol \text{ is not symmetric } :\Leftrightarrow
            \exists x, y \in \reluniverse: \rel{x}{y} \land \nrel{y}{x}$
          \label{def:no-symmetry}
    \item $\relsymbol \text{ is symmetric } :\Leftrightarrow
            \forall x, y \in \reluniverse: \rel{x}{y} \rightarrow \rel{y}{x}$
          \label{def:symmetry}
  \end{enumerate}
  \label{def:notions-of-symmetry}
\end{definition}

We will show that asymmetry implies both anti-symmetry and non-symmetry and is therefore a
"stronger" condition.
\begin{claim}
  Let $\reluniverse \neq \emptyset$ and $\relsymbol \neq \emptyset$.
  \[
    \relsymbol \text{ is asymmetric } \Rightarrow \relsymbol \text{ is anti-symmetric.}
  \]
\end{claim}

\begin{proof}
  Let $\relsymbol$ be asymmetric. Let $x, y \in \reluniverse$.
  If $\rel{x}{y}$ then $\nrel{x}{y}$ due to asymmetry of $\relsymbol$.
  Hence $\rel{x}{y} \land \nrel{x}{y}$ will never hold, making the implication in
  \ref{def:anti-symmetry} of definition \ref{def:notions-of-symmetry} vacuously true.

\end{proof}

\begin{claim}
  Let $\reluniverse \neq \emptyset$ and $\relsymbol \neq \emptyset$.
  \[
    \relsymbol \text{ is asymmetric } \Rightarrow \relsymbol \text{ is not symmetric.}
  \]
\end{claim}

\begin{proof}
  Let $\relsymbol$ be asymmetric. Let $x, y \in \reluniverse$.
  If $\rel{x}{y}$ then $\nrel{x}{y}$ due to asymmetry of $\relsymbol$.
  Hence, there exists $x, y \in \reluniverse$ s.t. $\rel{x}{y} \land \nrel{y}{x}$.

\end{proof}

\begin{claim}[Symmetry of the Empty Relation]
  Let $\reluniverse \neq \emptyset$ and $\relsymbol = \emptyset$.

  \begin{enumerate}
    \item $\relsymbol$ is asymmetric.
    \item $\relsymbol$ anti-symmetric.
    \item $\relsymbol$ is symmetric.
  \end{enumerate}
\end{claim}

\begin{proof}
  As all three properties are defined as logical implications, their antedecents will never hold because:
  \[
    \rel{x}{y} \Leftrightarrow (x, y) \in \relsymbol = \emptyset
  \]
\end{proof}

Let in the following $\reluniverse = \{1, 2, 3\}$
\begin{itemize}
  \item $\relsymbol = \{(1, 2), (2, 1)\}$ is symmetric but not anti-symmetric.
  \item $\relsymbol = \{(1, 2)\}$ is anti-symmetric but not symmetric.
  \item $\relsymbol = \{(1, 1)\}$ is both anti-symmetric and symmetric.
  \item $\relsymbol = \{(1, 2), (2, 1), (1, 3)\}$ is neither anti-symmetric nor symmetric.
\end{itemize}

